\documentclass[10pt]{article}

\usepackage{graphicx}
\usepackage{times}
\usepackage[T1]{fontenc}
\usepackage{amsmath}
\usepackage[ngerman]{babel}
\usepackage[ansinew]{inputenc}
\usepackage{fixltx2e}
\usepackage{setspace}
\usepackage{geometry}
\usepackage{amsthm}
\usepackage{cite}
%\usepackage{multirow}
\geometry{a4paper,left=30mm,right=30mm, top=3cm, bottom=3cm}
\newcommand{\result}[1]{\underline{\underline{#1}}}
%\usepackage{mathtools}
%\usepackage{dingbat}
%\usepackage{bbding}
\usepackage{amssymb}
\usepackage{amsfonts}
\newcommand{\ol}[1]{\overline{#1}}
\newcommand{\varrule}[2]{#1& $\longrightarrow$ & #2\\}
\newcommand{\predic}[2]{\mbox{\textit{#1}}(\mbox{\textit{#2}})}

\title{Methodenbeschreibung}
\author{Team 31 \& Team 16}

\begin{document}

\maketitle

\section{Allgemein}
	F�r jedes \texttt{private} Attribut gibt es falls n�tig eine Methode \texttt{get()} und \texttt{set(...)}, 
	welche der �bersicht und Offensichtlichkeit ihrer Funktion nicht weiter aufgef�hrt werden.
\section{Methodenbeschreibung}

%=============================================================
	\subsection{Modulhandbuch}
		\begin{tabbing}
			................................. \= . \kill
			METHODE \>\texttt{public void ModulInModulhandbuchAufnehmen(Modul aufzunehmendes)}\\
			BESCHREIBUNG \>F�gt \texttt{aufzunehmendes} zum Modulhandbuch hinzu.\\
		\end{tabbing}
		
	%=============================================================
	\subsection{Dokument}
		
		\begin{tabbing}
			................................. \= . \kill
			METHODE \>\texttt{public void DokumentAnzeigen()}\\
			BESCHREIBUNG \>Das Dokument wird im Webbrowser angezeigt. Speichern und Drucken dank PDF-Reader.\\
		\end{tabbing}
		
		\begin{tabbing}
			................................. \= . \kill
			METHODE \>\texttt{public void DokumentHerunterladen()}\\
			BESCHREIBUNG \>L�dt Dokument als PDF herunter. Der PDF-Reader liefert M�glichkeit zum Drucken und Speichern.\\
		\end{tabbing}
		
		\begin{tabbing}
			................................. \= . \kill
			METHODE \>\texttt{public void DokumentDrucken()}\\
			BESCHREIBUNG \>Der Druckdialog f�r das zu Grunde liegende Dokument wird angezeigt.\\
		\end{tabbing}
		
		\begin{tabbing}
			................................. \= . \kill
			METHODE \>\texttt{public void DokumentLoeschen()}\\
			BESCHREIBUNG \>L�scht Dokument aus der Datenbank, wenn es veraltet ist.\\
		\end{tabbing}
		
	%=============================================================
	\subsection{Modul}	
	
		\begin{tabbing}
			................................. \= . \kill
			METHODE \>\texttt{public static Modul ModulErstellen(string beschreibung)}\\
			BESCHREIBUNG \>Erstellt neues Modul mit Titel \texttt{beschreibung} und gibt es zur�ck.\\
		\end{tabbing}
		
		\begin{tabbing}
			................................. \= . \kill
			METHODE \>\texttt{public void ModulAendern(String beschreibung)}\\
			BESCHREIBUNG \>Setzt Modulbeschreibung neu.\\
		\end{tabbing}
		
		\begin{tabbing}
			................................. \= . \kill
			METHODE \>\texttt{public void ModulWeiterleiten()}\\
			BESCHREIBUNG \>Legt dem Koordinator Modul zur Pr�fung vor.\\
		\end{tabbing}
		
		\begin{tabbing}
			................................. \= . \kill
			METHODE \>\texttt{public void ModulVeroeffentlichen()}\\
			BESCHREIBUNG \>Setzt \texttt{status} auf �ffentlich.\\
		\end{tabbing}
		
		\begin{tabbing}
			................................. \= . \kill
			METHODE \>\texttt{public void ModulAblehnen()}\\
			BESCHREIBUNG \>Setzt \texttt{status} auf abgelehnt.\\
		\end{tabbing}
		
	%=============================================================
	\subsection{Benutzer}
	
		\begin{tabbing}
			................................. \= . \kill
			METHODE \>\texttt{public void EmailVersenden(Benutzer empfaenger, String text)}\\
			BESCHREIBUNG \>Sendet eine Nachricht mit Inhalt \texttt{text} an \texttt{empfaenger}.\\
		\end{tabbing}
	
		\begin{tabbing}
			................................. \= . \kill
			METHODE \>\texttt{public void JobsAbfragen()}\\
			BESCHREIBUNG \>Listet zu erledigende Aufgaben auf.\\
		\end{tabbing}
		
		\begin{tabbing}
			................................. \= . \kill
			METHODE \>\texttt{public void JobErledigen(int jobID)}\\
			BESCHREIBUNG \>Set status von des Jobs mit �bergebener \texttt{jobID} auf erledigt.\\
		\end{tabbing}
		
		\begin{tabbing}
			................................. \= . \kill
			METHODE \>\texttt{public void StellvertreterAendern(Benutzer neuerStellvertreter)}\\
			BESCHREIBUNG \>�ndert Stellvertreter auf \texttt{neuerStellvertreter}.\\
		\end{tabbing}
		
		\begin{tabbing}
			................................. \= . \kill
			METHODE \>\texttt{public void AccouteinstellungenAendern()}\\
			BESCHREIBUNG \>Er�ffnet M�glichkeit zum �ndern der Accounteinstellungen.\\
		\end{tabbing}
		
		\begin{tabbing}
			................................. \= . \kill
			METHODE \>\texttt{public static Benutzer Login(String benutzername, String passwort)}\\
			BESCHREIBUNG \>Sucht entsprechenden Nutzer aus der DB, pr�ft das Passwort und gibt den Nutzer zur�ck.\\
		\end{tabbing}
		
		\begin{tabbing}
			................................. \= . \kill
			METHODE \>\texttt{public static void Registrieren(Benutzer neuerNutzer)}\\
			BESCHREIBUNG \>Antrag auf Registration f�r Nutzer \texttt{neuerNutzer}.\\
		\end{tabbing}
		
		\begin{tabbing}
			................................. \= . \kill
			METHODE \>\texttt{public Benutzer BenutzerAbrufen()}\\
			BESCHREIBUNG \>Gibt diesen Nutzer zur�ck.\\
		\end{tabbing}
		
		\begin{tabbing}
			................................. \= . \kill
			METHODE \>\texttt{public void BenutzerLoeschen()}\\
			BESCHREIBUNG \>L�scht diesen Nutzer.\\
		\end{tabbing}
		
		\begin{tabbing}
			................................. \= . \kill
			METHODE \>\texttt{public void BenutzerSperren()}\\
			BESCHREIBUNG \>Deaktiviert diesen Nutzer.\\
		\end{tabbing}
		
		\begin{tabbing}
			................................. \= . \kill
			METHODE \>\texttt{public void BenutzerEntsperren()}\\
			BESCHREIBUNG \>Aktiviert diesen Nutzer.\\
		\end{tabbing}
		
		\begin{tabbing}
			................................. \= . \kill
			METHODE \>\texttt{public void BenutzerHinzufuegen()}\\
			BESCHREIBUNG \>F�gt diesen Nutzer der DB hinzu.\\
		\end{tabbing}
		
		\begin{tabbing}
			................................. \= . \kill
			METHODE \>\texttt{public void PasswortZuruecksetzen()}\\
			BESCHREIBUNG \>Setz das Passwort auf ein zuf�lliges Passwort und schickt e-Mail an Nutzer.\\
		\end{tabbing}
		
		\begin{tabbing}
			................................. \= . \kill
			METHODE \>\texttt{public void RolleAendern(Rolle neueRolle)}\\
			BESCHREIBUNG \>Setzt Rolle des Nutzers auf \texttt{neueRolle}.\\
		\end{tabbing}
		
	%=============================================================	
	\subsection{Archiv}
		
		\begin{tabbing}
			................................. \= . \kill
			METHODE \>\texttt{public void EmailVersenden(Benutzer empfaenger, String text)}\\
			BESCHREIBUNG \>Sendet eine Nachricht mit Inhalt \texttt{text} an \texttt{empfaenger}.\\
		\end{tabbing}
		
	%=============================================================
	\subsection{Kalender}
		\begin{tabbing}
			................................. \= . \kill
			METHODE \>\texttt{public List<Kalendereintrag> KalenderAbrufen(Date datum)}\\
			BESCHREIBUNG \>Ruft zu erledigende Aufgaben f�r den Monat nach \texttt{datum} auf.\\
		\end{tabbing}
		
		\begin{tabbing}
			................................. \= . \kill
			METHODE \>\texttt{public void KalendereintragHinzufuegen(Kalendereintrag neuerEintrag)}\\
			BESCHREIBUNG \>F�gt \texttt{neuerEintrag} dem Kalender hinzu.\\
		\end{tabbing}

\end{document}